%-------------------------------------------------------------------------------
%	SECTION TITLE
%-------------------------------------------------------------------------------
\cvsection{University Projects}


%-------------------------------------------------------------------------------
%	CONTENT
%-------------------------------------------------------------------------------
\begin{cventries}

\cventry
{Many-particle Physics course | \href{https://drive.google.com/file/d/1Gf_Eeht8ysXzUMMGgl5LR_zTFmUtr4z9/view?usp=drive_link}{Report}} % Organisation
{\href{}{Anderson Impurity Problem}} % Project
{Bhubaneswar} % Location
{Aug 2024 - Nov 2024} % Date(s)
{
	\begin{cvitems} % Description(s) of project
		\item{Under the supervision of \href{https://sites.google.com/site/workpagetemp}{Dr. Anamitra Mukherjee}}
		\item {Study of dynamics of electrons in metal doped with impurities}
		\item {Different Born approximations and Feynman diagrams}\\
	\end{cvitems}
}


\cventry
{NISER Open-Lab} % Organisation
{\href{}{Secure communication with Chua circuit}} % Project
{Bhubaneswar} % Location
{Aug 2024 - Nov 2024} % Date(s)
{
	\begin{cvitems} % Description(s) of project
		\item {Design and analysis of Chua circuit and study of the non-linear element constructed using op-amp as negative resistance}
		\item {Synchronization of two Chua circuits and study of its stability for different parameter values}
		\item {Use of synchronization to securely transmit digital and analog information by using chaos to encode information}\\
		%		\item {\textbf{Technical Skills:} Python with Pandas, matplotlib, Seaborn.}
		%		\item {\textbf{Soft Skills:} Presentation skills, Leadership, Teamwork, Logical Thinking.}
	\end{cvitems}
}

\cventry
{ML course | \href{https://github.com/smlab-niser/24cs460/tree/main/projects/Group1}{GitHub-repo}} % Job title  
{Machine learning for wavefunction dynamics} % Organisation
{Odisha, India} % Location
{Jan-March 2024} % Date(s)
{
	\begin{cvitems} % Description(s) of tasks/responsibilities
		\item{Reconstruction and prediction of the probability density of a quantum system using PINN and regression}
		\item{Report can be found \href{https://github.com/smlab-niser/24cs460/blob/main/projects/Group1/end-sem/final_report_cs460.pdf}{here}}\\
	\end{cvitems}
}

%\cventry
%{NISER} % Job title 
%{Experimental verification of the Einstein-de Haas effect } % Organisation
%{Odisha, India} % Location
%{June 2023} % Date(s)
%{
%	\begin{cvitems} % Description(s) of tasks/responsibilities
%		\item{ A physical phenomenon in which a change in the magnetic moment of a free body causes the body to rotate following conservation of angular momentum}
%	\end{cvitems}\\
%}

\cventry
{NISER | Computational Physics Course} % Organisation
{\href{https://github.com/pritipriya-dasbehera/QNN_CP}{Quantum Neural Network}} % Project
{Bhubaneswar} % Location
{Aug 2023 - Oct 2023} % Date(s)
{
	\begin{cvitems} % Description(s) of project
		\item {Building a Image classifier algorithm with classic Convolutional Neural Network and a linear quantum layer}
		\item {Implementation of forward and backward pass on a quantum layer}\\
%		\item {\textbf{Technical Skills:} Python with Pandas, matplotlib, Seaborn.}
%		\item {\textbf{Soft Skills:} Presentation skills, Leadership, Teamwork, Logical Thinking.}
	\end{cvitems}
}

\cventry
{National Anveshika Experimental Skill Test} % Job title 
{Experimental verification of the Einstein-de Haas effect } % Organisation
{Odisha, India} % Location
{June 2023} % Date(s)
{
	 Einstein-de Haas effect is a physical phenomenon in which a change in the magnetic moment of a free body causes the body to rotate following conservation of angular momentum.
%	\begin{cvitems} % Description(s) of tasks/responsibilities
%		\item{ A physical phenomenon in which a change in the magnetic moment of a free body causes the body to rotate following conservation of angular momentum}
%	\end{cvitems}
}
%%---------------------------------------------------------
%  \cventry
%    {University of Leeds} % Organisation
%    {IPL Analysis} % Project
%    {Leeds, UK} % Location
%    {Nov 2021 - Dec 2021} % Date(s)
%    {
%      \begin{cvitems} % Description(s) of project
%        \item {Analysing data from 2008 to 2015 to discover patterns such as trends, correlations, and probabilities.}
%        \item {Finding the differences between the best teams and players in various fields, as well as their performance on the field and the opponents.}
%        \item {\textbf{Technical Skills:} Python with Pandas, matplotlib, Seaborn.}
%        \item {\textbf{Soft Skills:} Presentation skills, Leadership, Teamwork, Logical Thinking.}
%      \end{cvitems}
%    }
%
%%---------------------------------------------------------
%  \cventry
%    {University of Leeds} % Organisation
%    {Study of the behavior of Serial Killers} % Project
%    {Leeds, UK} % Location
%    {Oct 2021 - Dec 2021} % Date(s)
%    {
%      \begin{cvitems} % Description(s) of project
%        \item {Study of serial killers behavior with various motives such as convenience (did not want children or spouse), enjoyment, power, escape, or avoiding arrest.}
%        \item {Finding patterns in their starting and ending ages (when they are caught) and other factors of killing, as well as how it varies with different motives.}
%        \item {\textbf{Technical Skills:} R with ggplot2, tidyr, R Markdown.}
%        \item {\textbf{Soft Skills:} Report writing, Logical Thinking, Critical Thinking.}
%      \end{cvitems}
%    }

%---------------------------------------------------------
\end{cventries}
